\begin{song}{Busted}
  {} % Bruges ikke, lad stå blank
  {Melodi} % Titel, Kunstner - eks.: "Jutlandia, Kim Larsen". Hvis sangen er på sin egen melodi, brug da \SBOrgMel.
  {Forfatter} % Navnet på forfatteren. Undlad kaldenavne. Brug gerne TBF. Brug "&" frem for "og". Hvis forfatter er ukendt, lad da stå tom.
  {Anledning og år} % Eks. "Fysikrevy, 2010" eller "2010"
  {\NotCCLIed} % Lad stå som den er

  \begin{SBVerse}
    % Skriv vers her
  \end{SBVerse}

  \begin{SBChorus}
    % Skriv omkvæd her
  \end{SBChorus}

  \begin{SBSection*}
    % Skriv sektioner her. Hvis du ønsker lidt mellemrum for at give luft i et langt afsnit el.lign., brug da \\\medskip
  \end{SBSection*}
\end{song}

% 1. Stille, stille, stille, jeg har tømmermænd og min hjerne er vist svundet
% det sidste jeg så var en lunken Top
% som blev for hurtigt bundet.
% 2. Det hyler og tuder i min hjerne nu væk er spiritussens charmer
% på gaden kører bilerne hurtigt forbi og fuglene de larmer.
% Omkvæd:
% Jeg’r busted
% åh, busted
% jeg har brækket mig i cisternerne og ønsker mig langt ud i det blå.
% 3. Stille, stille, stille, jeg har tømmermænd der er larm, selv under dynen
% jeg strækker mig efter en doven Top
% og putter den i trynen
% 4. I aftes gjord’ jeg noget som man ikke skal
% man skal nemlig ikke tylle
% til sidst sag’ jeg dav til en stor og mægtig spand som jeg næsten ku’ fylde
% Omkvæd:
% Jeg’r busted
% åh, busted
% Jeg tog vist en pige på lanternerne min kind er nu langt ud i det blå
% Omkvæd:
% Jeg’r busted
% åh, busted
% Som sprunget ud af et mareridt
% tager tømmeren hævn, og han kræver sit åh-nej. . .
% Jeg ønsker ham langt ud i det blå
% Jeg’r busted – åh, busted.
\begin{song}{Titel}
  {} % Bruges ikke, lad stå blank
  {Melodi} % Titel, Kunstner - eks.: "Jutlandia, Kim Larsen". Hvis sangen er på sin egen melodi, brug da \SBOrgMel.
  {Forfatter} % Navnet på forfatteren. Undlad kaldenavne. Brug gerne TBF. Brug "&" frem for "og". Hvis forfatter er ukendt, lad da stå tom.
  {Anledning og år} % Eks. "Fysikrevy, 2010" eller "2010"
  {\NotCCLIed} % Lad stå som den er

  \begin{SBVerse}
    % Skriv vers her
  \end{SBVerse}

  \begin{SBChorus}
    % Skriv omkvæd her
  \end{SBChorus}

  \begin{SBSection*}
    % Skriv sektioner her. Hvis du ønsker lidt mellemrum for at give luft i et langt afsnit el.lign., brug da \\\medskip
  \end{SBSection*}
\end{song}

% \beginsong{Laborant}
% [sr={Melodi:Nøddepatruljen}
% ,by={MBK-revyen 2013}
% ]

% \beginverse
% Gir geler, problemer,
% Og sejler dit projekt
% Dit array, er gay,
% Din protokol er væk
% \endverse


% \beginverse
% Ja så kommer de og redder dig
% Specialet blir' en leg!
% La-la-la-la-bo-rant
% De kan mixe
% La-la-la-la-bo-rant
% Uden at kikse!
% For hvis du er I nød så kommer de,
% bar' husk at spørg før de tager fri!
% \endverse


% \beginverse
% De blander, buffer,
% I hver koncentration
% Men kommer, aldrig,
% på en publikation!
% \endverse


% \beginverse
% Oprens mit plasmid, bestil enzym
% De er et sørg'ligt syn!
% La-la-la-la-bo-rant
% De en gave
% La-la-la-la-bo-ra-
% -Torieslave!
% \endverse


% \beginverse
% For de er svar på hver professors bøn,
% det godt de får så lidt i løn!
% La-la-la-la-bo-rant
% De en gave
% La-la-la-la-bo-ra-
% -Torieslave!
% \endverse


% \beginverse
% For de er svar på hver professors bøn ...
% \endverse


% \beginverse
% La-la-la-la-bo-rant!
% \endverse

% \endsong

\begin{song}{$\rho$-$\phi$-$\mu$}
  {} % Bruges ikke, lad stå blank
  {Do Re Mi, Sound of Music} % Titel, Kunstner - eks.: "Jutlandia, Kim Larsen". Hvis sangen er på sin egen melodi, brug da \SBOrgMel.
  {Ole Søe Sørensen \& Søren Gammelmark} % Navnet på forfatteren. Undlad kaldenavne. Brug gerne TBF. Brug "&" frem for "og". Hvis forfatter er ukendt, lad da stå tom.
  {\TKET, 2010} % Eks. "Fysikrevy, 2010" eller "2010"
  {\NotCCLIed} % Lad stå som den er

  \begin{SBVerse}
    Lad os starte helt fra starten\\
    af det græske alfabet.\\
    Når du tæller, du starter med 1-2-3, i fysik vi starter med omega\\
    $\omega$ \emph{($\omega$)}\\
    Ja først som sidst er\\
    $\omega$ \emph{($\omega$)}\\
    $\omega$, $\mu$, $\psi$, $\phi$, $\chi$, $\pi$, $\xi$
  \end{SBVerse}

  \begin{SBVerse}
    $\omega$ er en frekvens,\\
    $\epsilon$ er ikk’ så stor,\\
    $\mu$ er ikke særklig langt!\\
    $\phi$ er noget med Niels Bohr,\\
    $\iota$, den gør ikk’ en ski’,\\
    $\lambda$ er en egenværdi,\\
    $\xi$ den ska’ du hold’ dig fra, \\
    $\kappa$ bringer os tilbag’ til $\omega$
  \end{SBVerse}

  \begin{SBVerse}
    på $\nu$ er en frekvens,\\
    $\eta$ er en krum metrik,\\
    $\upsilon$, der lyder svensk,\\
    $\gamma$, den kan tiden stræk’,\\
    Vinkler måler man med $\phi$,\\
    $\theta$ kan vi alle li’,\\
    $\pi$ har vist nok en værdi,\\
    det er mange tegn her i det græske
  \end{SBVerse}

  \begin{SBVerse}
    $\alpha$, $\beta$, det er tal,\\
    $\rho$, det er en densitet,\\
    $\chi$ – når du er træt af x,\\
    $\tau$ er tiden, sid’n det sket’,\\
    $\zeta$ er en syg funktion,\\
    $\sigma$ en permutation,\\
    Ingen bruger o,\\
    Nu kan I $\delta$ i TØ!
  \end{SBVerse}

  \begin{SBSection*}
    At man alfabetet ka',\\
    er $\alpha$ og $\omega$
  \end{SBSection*}

\end{song}
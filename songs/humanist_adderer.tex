\begin{song}{Når en humanist adderer}
  {} % Bruges ikke, lad stå blank
  {Peberkagesangen, Dyrene i Hakkebakkeskoven} % Titel, Kunstner - eks.: "Jutlandia, Kim Larsen". Hvis sangen er på sin egen melodi, brug da \SBOrgMel.
  {TÅGEKAMMERET} % Navnet på forfatteren. Undlad aliasser. Brug "&" frem for "og". Hvis forfatter er ukendt, lad da stå tom.
  {} % Eks. "Fysikrevy 2010" eller "2010"
  {\NotCCLIed} % Lad stå som den er

  \begin{SBVerse}
    Når en humanist adderer,\\
    tar’ hun først et $\pi$ i fjerde,\\
    ganger det med $n$ matricer,\\
    mens hun vælger en "sød" brøk.\\
    Læg det til determinanten\\
    det er humanistisk fjanten -\\
    for hun når jo aldrig læng’re\\
    end at svaret det er $x$.
  \end{SBVerse}

  \begin{SBVerse}
    Når en biolog skal lære,\\
    hvordan man multiplicerer,\\
    skal eksempler observeres,\\
    hvor to køer bli’r til tre.\\
    Og gør man det en gang mere,\\
    Så bli’r tre kø’r jo til flere!\\
    Så hun kender kun’ et svar:\\
    At der er flere nu end før.
  \end{SBVerse}

  \begin{SBVerse}
    Geologer de kan regne,\\
    dem skal man ikke forklejne.\\
    De bestemmer snildt en alder,\\
    når de måler på en sten.\\
    Men har deres kompetencer\\
    nogen sociale nuancer?\\
    Man skal være i en grusgrav\\
    før det bliver relevant!
  \end{SBVerse}

  \begin{SBVerse}
    Men hos os fra TÅGEKAMMERET\\
    giver regning ingen jamren,\\
    vores ligninger gi’r mening\\
    også med socialt aspekt.\\
    Vi beregner øl i kasser\\
    med potens og inderklasser,\\
    og vor KA\$\$ har altid styr på,\\
    hvem der skylder hvad for hvad.
  \end{SBVerse}

  \begin{SBVerse}
    Så lektionen den må være:\\
    Hold mat/fys’erne i ære,\\
    de har styr på deres sager,\\
    og de drikker mange øl. [Skål!]\\
    Udregningerne skal komme\\
    fra han-dyrene med vomme,\\
    der beviser de har været\\
    alt for meget på TK!
  \end{SBVerse}
\end{song}
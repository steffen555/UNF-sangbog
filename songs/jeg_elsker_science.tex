\begin{song}{Jeg elsker "science"}
  {} % Bruges ikke, lad stå blank
  {Melodi} % Titel, Kunstner - eks.: "Jutlandia, Kim Larsen". Hvis sangen er på sin egen melodi, brug da \SBOrgMel.
  {Forfatter} % Navnet på forfatteren. Undlad kaldenavne. Brug gerne TBF. Brug "&" frem for "og". Hvis forfatter er ukendt, lad da stå tom.
  {Anledning og år} % Eks. "Fysikrevy, 2010" eller "2010"
  {\NotCCLIed} % Lad stå som den er

  \begin{SBVerse}
    % Skriv vers her
  \end{SBVerse}

  \begin{SBChorus}
    % Skriv omkvæd her
  \end{SBChorus}

  \begin{SBSection*}
    % Skriv sektioner her. Hvis du ønsker lidt mellemrum for at give luft i et langt afsnit el.lign., brug da \\\medskip
  \end{SBSection*}
\end{song}

% Først så var jeg bange, jeg var skrækslagen
% Jeg tænkte, det lød lig’så skrækkeligt som et stræklagen.
% Og jeg sad søvnløs hele natten
% stirred’ tomt ind i min pejs
% Før hed det naturvidenskab,
% nu skal man sige “science”

% Syn’s det var dumt - noget rigtigt lort
% Men nu der ingen vej tilbage, “science” er det nye sort.
% Jeg sku’ ha sagt noget eller gjort noget, 
% åh gået i protest
% Men sket er sket, og science er jo dansk når det er bedst

% Det’ ikk’ så slemt, og egentlig nemt
% Syv bogstaver?
% Det enkelt og bekvemt
% Hvorfor skal man slæbe rundt på gamle danske ord?
% Alle snakker engelsk nu, 
% ligegyldigt hvor de bor.

% Jeg elsker science, jeg elsker science!
% Fysik, kemi, biologi 
% det er jo alt for nice.
% Matematik, geologi 
% og medicinsk teknologi, jeg elsker science
% Jeg elsker science!

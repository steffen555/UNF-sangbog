\begin{song}{*NULL}
  {} % Bruges ikke, lad stå blank
  {Den røde tråd, Shu-bi-dua} % Titel, Kunstner - eks.: "Jutlandia, Kim Larsen". Hvis sangen er på sin egen melodi, brug da \SBOrgMel.
  {Troels + Signe, Sven, Niels} % Navnet på forfatteren. Undlad kaldenavne. Brug gerne TBF. Brug "&" frem for "og". Hvis forfatter er ukendt, lad da stå tom.
  {DIKUrevy, 2018} % Eks. "Fysikrevy, 2010" eller "2010"
  {\NotCCLIed} % Lad stå som den er

\begin{SBSection*}
  Hvad mon der er \\
  I det der hul \\
  Når man derefererer NULL? \\
  En tredje bit? \\
  Nog'l søgetræ'r? \\
  Mon Naurs løsen ligger der? \\
\end{SBSection*}
  
\begin{SBVerse}
  Rundt i lagret ku' jeg hop' \\
  Indtil kernen sagde ``stop! \\
  Din peger peger på for lidt \\
  Du har slet ingen bits før 1.'' \\ \medskip
\end{SBVerse}  

\begin{SBVerse}
  Til EDB der er jeg vaks \\
  Så til Linus klag'de jeg straks \\
  ``Betalte for en hel' maskin' \\
  Men pladsen NULL er ikke min!'' \\
\end{SBVerse}

\begin{SBChorus}
  Jeg koded' og roded' \\
  Fik castet for- og baglæns \\
  I Compsys lærte jeg \\
  at NULL betyder fejl \\ \medskip
  
  Men det er bare ik' O-K \\
  (når) viden kun er for de få \\
  (Hvis) ik' de ville hem'lighold' \\
  måt' jeg vel gerne læs' fra NULL \\ \medskip
  
  Jeg hacked' og cracked' \\
  I PCS der sagd' de \\
  ``læs NULL og så' du færdig'' \\
  Hvor galt kan det dog gå? \\
\end{SBChorus}

\medskip

\begin{SBSection*}
  For hvad kan der vær' \\
  Hem'ligt i NULL? \\
  Hvorfor må jeg ik' få kontrol? \\
  Hvad vil de ikke ha' vi ser? \\
  Der må for fa'n da være mer'! \\
\end{SBSection*}

\end{song}
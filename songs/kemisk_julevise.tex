\begin{song}{Kemisk julevise}
  {} % Bruges ikke, lad stå blank
  {Højt fra træets grønne top} % Titel, Kunstner - eks.: "Jutlandia, Kim Larsen". Hvis sangen er på sin egen melodi, brug da \SBOrgMel.
  {} % Navnet på forfatteren. Undlad kaldenavne. Brug gerne TBF. Brug "&" frem for "og". Hvis forfatter er ukendt, lad da stå tom.
  {} % Eks. "Fysikrevy, 2010" eller "2010"
  {\NotCCLIed} % Lad stå som den er

  \begin{SBVerse}
    Lehninger og Holleman\\
    kan vi uden skelen\\
    alle ved, at H$_2$O er vand,\\
    læg jer nu i Se.\\
    Jule-Tin-een falder blidt,\\
    snart er jorden hvid som CaCO$_3$,\\
    smuk-R$_2$CO-er spiller,\\
    spurven slår R-CN-ler.
  \end{SBVerse}

  \begin{SBVerse}
    Nu hvor vi om dette B'd,\\
    alle blevet m-C$_2$H$_5$OC$_2$H$_5$,\\
    sk-R-NH$_2$ sang i lystigt kor\\
    gøre maven letter.\\
    Vi en m-CH$_3$CO-led' ned,\\
    ROH gi'r h-Fe-en fred.\\
    Titan gla-Ag-i tømte\\
    Na$_2$CO$_3$-H$_2$O forsømte.
  \end{SBVerse}

  \begin{SBVerse}
    R$_1$COOR$_2$ hun har ingen SCN$_2$ -\\
    ...ser rundt med CnH$_2$n -\\
    ...der gaven ka-NO$_2$\\
    N$_2$ til en kjole.\\
    Højt Mn-tes uafbrudt\\
    og med øjet slår Bi.\\
    Højt en m-R-CH(OR)2'er\\
    hoved-C$_{10}$H$_{16}$ maler.
  \end{SBVerse}

  \begin{SBVerse}
    Aluminium Sn-g det er nu spist op,\\
    og man det Ti-er,\\
    som en W i sin krop\\
    mad for vore ganer.\\
    Jern-sten ku' vi alle Lithium\\
    den var så HgCl$_2$ vi\\
    Ni-et og C$_{12}$H$_{22}$O$_{11}$\\
    før vi lyset slukker.
  \end{SBVerse}
\end{song}
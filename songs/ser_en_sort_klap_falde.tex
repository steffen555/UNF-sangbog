\begin{song}{Når jeg ser en sort klap falde}
  {} % Bruges ikke, lad stå blank
  {Naar jeg ser et rødt Flag smælde, Oskar Hansen} % Titel, Kunstner - eks.: "Jutlandia, Kim Larsen". Hvis sangen er på sin egen melodi, brug da \SBOrgMel.
  {Michael Bisgaard Olesen} % Navnet på forfatteren. Undlad kaldenavne. Brug gerne TBF. Brug "&" frem for "og". Hvis forfatter er ukendt, lad da stå tom.
  {\TKET{}, 2008} % Eks. "Fysikrevy, 2010" eller "2010"
  {\NotCCLIed} % Lad stå som den er

  \begin{SBVerse}
    Når jeg ser en sort klap falde\\
    til en eksamination,\\
    kan jeg høre de gamle guder kalde\\
    og så træder jeg sgu i aktion.\\
    Du kan råbe og skrige og græde,\\
    men det hjælper dig ikke en kaj,\\
    for jeg er i mit es når jeg griner:\\
    “du sku’ vist have forberedt dig!”
  \end{SBVerse}

  \begin{SBVerse}
    Ja, jeg er den syge censor\\
    jeg kan ikke styre mig.\\
    Mens du forbereder sidder jeg og venter\\
    på at spolere eksamen for dig.\\
    Når du så kommer ind i lokalet\\
    så begynder jeg at pil’ dig ned.\\
    For eksempel ku’ jeg start’ med at sige:\\
    “du er sgu da urimeligt fed!”
  \end{SBVerse}

  \begin{SBVerse}
    Når du så er brudt helt sammen\\
    og ikk aner levende råd,\\
    vil jeg puste lidt mere til flammen,\\
    helmer ikke, før jeg hører gråd.\\
    Det kan vær’ at jeg spør’ ekstra ind til\\
    det der emne du fuckede op i.\\
    Ellers finder jeg da bar’ på noget sværre,\\
    det gi’r seancen lidt mere krydderi.
  \end{SBVerse}

  \begin{SBVerse}
    Efter femogfyr’ minutter\\
    kan den ikke trækkes mer’,\\
    jeg kan mærke en æra der slutter,\\
    men så skal jeg gi’ dig karakter!\\
    Det er næsten det bedst’ at det hele\\
    når jeg gi’r dig et rundt minus tre,\\
    jeg kan slet ikke skjule min glæde\\
    når jeg skrig’r “det war hwa’ det ku’ blyw’ til!”
  \end{SBVerse}
\end{song}

\begin{song}{Nano}
  {} % Bruges ikke, lad stå blank
  {Melodi} % Titel, Kunstner - eks.: "Jutlandia, Kim Larsen". Hvis sangen er på sin egen melodi, brug da \SBOrgMel.
  {Forfatter} % Navnet på forfatteren. Undlad kaldenavne. Brug gerne TBF. Brug "&" frem for "og". Hvis forfatter er ukendt, lad da stå tom.
  {Anledning og år} % Eks. "Fysikrevy, 2010" eller "2010"
  {\NotCCLIed} % Lad stå som den er

  \begin{SBVerse}
    % Skriv vers her
  \end{SBVerse}

  \begin{SBChorus}
    % Skriv omkvæd her
  \end{SBChorus}

  \begin{SBSection*}
    % Skriv sektioner her. Hvis du ønsker lidt mellemrum for at give luft i et langt afsnit el.lign., brug da \\\medskip
  \end{SBSection*}
\end{song}

% 1. Nano, Nano.
% Har du set en Nano? Fremtidens genier
% kan stå samlet på en tier
% 2. Nano, Na-na-na-Nano.
% Er det ikke sandt? Jo!
% Venlige mikrober.
% De er så små, at man ikke
% kan se dem - selv i mikroskoper.
% Omkvæd:
% Nano. Vi elsker jer
% I gør det hele letter’.
% For I kan bygge fremtidens tabletter (Na-na-na-no-na-no-na-no-na-no) med atomer og nano-pincetter.
% 3. Nano, Nano. Søde lille Nano. Bittesmå profeter på 10−9 meter.
% 4. Nano. Na-na-na-Nano.
% Vogt dig for en Nano. Fremtidens spioner.
% De infiltrerer hvad som helst ved hjælp af simple diffusioner.
% Omkvæd:
% 5.
% Nano. Vi elsker jeres
% nuttede studiner.
% For de kan lave bittesmå maskiner (Na-na-na-no-na-no-na-no-na-no) som kan vaccinere selv vacciner.
% Nano, åh nano!
% Ingen kan som Nano nanorere sproget:
% Med nano-adaptive nano- gloser kører Nanotoget.
% Omkvæd:
% Nano, Vi elsker jer!
% Mikro er blot et minde.
% Og selvom I er ret svære at finde (Be-Be-Besenbacher-Besenbacher) ved vi, Nano-tiden den er inde.
\begin{song}{Terminalsangen}
  {} % Bruges ikke, lad stå blank
  {Vuffelivov, Shu-bi-dua} % Titel, Kunstner - eks.: "Jutlandia, Kim Larsen". Hvis sangen er på sin egen melodi, brug da \SBOrgMel.
  {} % Navnet på forfatteren. Undlad kaldenavne. Brug gerne TBF. Brug "&" frem for "og". Hvis forfatter er ukendt, lad da stå tom.
  {DIKUrevy, 1978} % Eks. "Fysikrevy, 2010" eller "2010"
  {\NotCCLIed} % Lad stå som den er

  \begin{SBVerse}
Jeg har en skærm med mange taster\\
En for hvert symbol\\
Og bagved sidder lysintensiteten\\
Den ledning har mange tråde\\
En til hver sin bit\\
og en ekstra en til pariteten\\
Når man har venner og kærester, så er man normal\\
Men de ta'r tiden fra mig og min terminal
  \end{SBVerse}

  \begin{SBVerse}
Jeg er koblet via DIXI\\
Når DIXI ellers vil\\
Og der er plads på centrets multiplekser\\
Når jeg har lyst så kan jeg sidder\\
Og lege natten lang\\
Med RECKUs mange programmelkomplekser\\
Når man har venner og kærester så er man normal\\
Men de ta'r tiden fra mig og min terminal
  \end{SBVerse}

  \begin{SBVerse}
Jeg kør' på en maskine\\
Der klarer tusind jobs\\
Selvom deta'r syv lange og syv bredde\\
CAU'en har den to af\\
Og det er vældig smart\\
En til hvis den anden sku' vær' nede\\
Når man har venner og kærester så er man normal\\
Men de ta'r tiden fra mig og min terminal
  \end{SBVerse}

  \begin{SBVerse}
Jeg spiller skak og kryds og bolle\\
Den hele lange nat\\
Det er nu trist man ingen kender\\
For selvom den er dejlig\\
Så er den datamat\\
Nu kun et surrogat for menn'ske-venner\\
Når man har venner og kærester så er man normal\\
Og har det bedre end mig med min terminal
  \end{SBVerse}
\end{song}
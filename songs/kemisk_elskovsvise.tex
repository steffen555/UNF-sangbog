\begin{song}{Kemisk elskovsvise}
  {} % Bruges ikke, lad stå blank
  {Melodi} % Titel, Kunstner - eks.: "Jutlandia, Kim Larsen". Hvis sangen er på sin egen melodi, brug da \SBOrgMel.
  {Forfatter} % Navnet på forfatteren. Undlad kaldenavne. Brug gerne TBF. Brug "&" frem for "og". Hvis forfatter er ukendt, lad da stå tom.
  {Anledning og år} % Eks. "Fysikrevy, 2010" eller "2010"
  {\NotCCLIed} % Lad stå som den er

  \begin{SBVerse}
    % Skriv vers her
  \end{SBVerse}

  \begin{SBChorus}
    % Skriv omkvæd her
  \end{SBChorus}

  \begin{SBSection*}
    % Skriv sektioner her. Hvis du ønsker lidt mellemrum for at give luft i et langt afsnit el.lign., brug da \\\medskip
  \end{SBSection*}
\end{song}

% \beginsong{Kemisk elskovsvise}[sr={Melodi: Santa Lucia}
% ,
% by={}
% ,
% cr={}]
% \beginverse
% Oh Pige, vær mig huld,
% fattig på gods og Au,
% står her din riddersmand - 
% går gennem ild og H$_2$O,
% for dig jeg ofrer alt,
% du er mig livets NaCl,
% smiler du til verdens vrimmel,
% Al$_2$O$_3$ og himmel.

% \endverse
% \beginverse

% Sødeste lille skalk,
% tag fra mig længsels CaO,
% sig blot et kærligt ord,
% håbet i hjertet B.
% I dine øjne fandt,
% jeg livets Cx,
% helt til jeg mit øje lukker,
% for dig jeg C$_{12}$H$_{22}$O$_{11}$.

% \endverse
% \beginverse

% Bi du længer står,
% får jeg mit banesår,
% er da mit ønske galt,
% skal håbet Na$_3$SbS$_4$$\cdot$9H$_2$O.
% Må jeg for NaOH og H$_2$O
% vandre i ensom stand,
% til en P vist jeg haster,
% og ned mig kaster.
% \endverse
% \endsong
